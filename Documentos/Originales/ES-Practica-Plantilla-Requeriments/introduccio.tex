\section{Introducció}\label{sec:intro}


La idea es una especie de Escape Room. Los jugadores tendrán que irse pasando pruebas por cada habitación que se encuentren (niveles). Al pasarse la prueba, podrán acceder a la siguiente pantalla. El ranking va por tiempo y nivel conseguido. En las habitaciones pueden haber aleatoriamente objetivos “bonus” que al conseguirlos permitirían al jugador saltarse niveles o pistas, reducir el tiempo final de la prueba, etc.
Habrá pruebas contrarreloj aleatoriamente que obligarán al jugador a pasarla en el tiempo establecido. En caso de no conseguirlo, la partida terminará y deberán volver a empezar.
El juego es individual. No puedes morir, a menos que te encuentres en una sala contrarreloj. 
Las puntuaciones serán individuales, pero se almacenarán de forma online, es decir, todo el mundo podrá ver los resultados de los demás.
En el juego existirá el rol de administrador que permitirá gestionar los niveles y tener privilegios extras.
El progreso de los usuarios no se perderá, es decir, la próxima vez que vuelvan podrán continuar por donde lo dejaron.


\begin{itemize}
\item  \textbf{BLABLABLA}
\item  \textit{\textbf{BLABLABLA}}.
\end{itemize}

En la secció \ref{sect:RNF} blablabla \newline \newline

En les següents seccions es fan servir els següents assumpcions:

En relació a les \textbf{Prioritats}:

\begin{itemize}
\item  \textbf{A} : Prioritat Alta.
\item  \textbf{M} : Prioritat Mitja
\item  \textbf{B} : Prioritat Baixa.
\end{itemize}

En relació a les \textbf{Verificacions}:

\begin{itemize}
\item  \textbf{R} : Verificació per Revisió.
\item  \textbf{T} : Verificació per Test.
\item  \textbf{D} : Verificació per Disseny.
\end{itemize}
