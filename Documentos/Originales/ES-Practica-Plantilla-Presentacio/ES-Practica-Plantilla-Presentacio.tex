\documentclass{beamer}
\usepackage{beamerthemesplit} % new

\usepackage[utf8]{inputenc}

\begin{document}
\title{NOM Del Joc}
\author{ES\_2019\_GXX-Y-NN}
\date{\today}

\frame{\titlepage}

\frame{\frametitle{Table of contents}\tableofcontents}


\section{Membres del equip}
\frame{\frametitle{Team}
  \begin{itemize}
    \item STUDENT 1
    \item STUDENT 2
    \item STUDENT 3
    \item STUDENT 4
    \item STUDENT 5
  \end{itemize}
}

\section{Joc}
\frame{\frametitle{Descripció}
}

\section{Jira}
\subsection{Product Backlog/Tasques}
\frame{\frametitle{Product Backlog}
}

\frame{\frametitle{Tasques}
}

\subsection{Burndown chart}
\frame{\frametitle{Burndown chart}

  \begin{figure}[ht]
  \centering
  \includegraphics[width=1.1\textwidth]{./imatges/burndown-tutorial_02.png}
  \label{fig:bdchart}
\end{figure}

}

\section{Bitbucket}
\frame{\frametitle{Bitbucket}
}

\section{Especificacions}
\subsection{Requeriments}
\frame{\frametitle{Requeriments}
}

\subsection{Casos d'ús}
\frame{\frametitle{Diagrama}

  \begin{figure}[ht]
  \centering
  \includegraphics[width=1.1\textwidth]{./imatges/usecase.png}
  \label{fig:usecase}
   \end{figure}
}

\frame{\frametitle{Casos d'ús (especificacions)}
}

\section{Conclusions}
\frame{\frametitle{Conclusions}
}

\end{document}
