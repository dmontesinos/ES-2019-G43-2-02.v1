\section{Caso de uso : 0001 - Identificación de usuario}\label{sec:uc0}



\subsection{Flujo de eventos}

\subsubsection{Básico}

\begin{enumerate}
\item Paso 1
El usuario ingresa sus datos
\item Paso 2
Si el login es correcto, se accede al sistema, sino muestra un error.
\end{enumerate}

  \subsubsection{Alternativo}
En caso de muchos errores se bloqueará la cuenta durante unos minutos.
\subsection{Precondiciones}
\begin{itemize}
\item No hay precondiciones.
\end{itemize}
\subsection{Postcondiciones}
\begin{itemize}
\item Se redirige al usuario al menú principal.
\end{itemize}
\section{Caso de uso : 0002 - ADMIN: creación y modificación de mapas}\label{sec:uc0}



\subsection{Flujo de eventos}

\subsubsection{Básico}

\begin{enumerate}
\item Paso 1
El administrador seleccionará la opción de modificar mapa.
\item Paso 2
El sistema le mostrará los mapas disponibles.
\item Paso 3
El administrador seleccionará el mapa a modificar.
\item Paso 4
El sistema enviará al administrador al editor de mapas.
\end{enumerate}


\subsection{Precondiciones}
\begin{itemize}
\item No hay precondiciones.
\end{itemize}
\subsection{Postcondiciones}
\begin{itemize}
\item Se redirige al usuario al menú principal.
\end{itemize}

\section{Caso de uso : 0003 - ADMIN: Eliminar cuentas}\label{sec:uc0}



\subsection{Flujo de eventos}

\subsubsection{Básico}

\begin{enumerate}
\item Paso 1
El sistema mostrará las cuentas al administrador.
\item Paso 2
El administrador seleccionará la cuenta a eliminar.
\item Paso 3
El sistema borrará la cuenta.
\end{enumerate}

\subsection{Precondiciones}
\begin{itemize}
\item No hay precondiciones.
\end{itemize}
\subsection{Postcondiciones}
\begin{itemize}
\item Se redirige al usuario al menú principal.
\end{itemize}

\section{Caso de uso : 0004 - ADMIN: Crear cuenta}\label{sec:uc0}



\subsection{Flujo de eventos}

\subsubsection{Básico}

\begin{enumerate}
\item Paso 1
El sistema pedirá los datos para la nueva cuenta.
\item Paso 2
El administrador especificará todas las características de la cuenta.
\item Paso 3
El sistema creará la nueva cuenta.
\end{enumerate}

  \subsubsection{Alternativo}
En caso de existir la cuenta, el sistema avisará con un error.
\subsection{Precondiciones}
\begin{itemize}
\item No hay precondiciones.
\end{itemize}
\subsection{Postcondiciones}
\begin{itemize}
\item Se redirige al usuario al menú principal.
\end{itemize}
\section{Caso de uso : 0005 - USUARIO: Empezar partida}\label{sec:uc0}



\subsection{Flujo de eventos}

\subsubsection{Básico}

\begin{enumerate}
\item Paso 1
El usuario seleccionará la opción desde el menú principal.
\item Paso 2
El sistema creará la nueva partida.
\end{enumerate}
\subsection{Precondiciones}
\begin{itemize}
\item No hay precondiciones.
\end{itemize}
\subsection{Postcondiciones}
\begin{itemize}
\item Se redirige al usuario al menú principal.
\end{itemize}
\section{Caso de uso : 0006 - USUARIO: Borrar su partida}\label{sec:uc0}



\subsection{Flujo de eventos}

\subsubsection{Básico}

\begin{enumerate}
\item Paso 1
El usuario seleccionará la opción.
\item Paso 2
El sistema mostrará los mapas de ese usuario.
\item Paso 3
El usuario seleccionará un mapa
\item Paso 4
El sistema borrará el mapa.
\end{enumerate}

\subsection{Precondiciones}
\begin{itemize}
\item No hay precondiciones.
\end{itemize}
\subsection{Postcondiciones}
\begin{itemize}
\item Se redirige al usuario al menú principal.
\end{itemize}



\section{Caso de uso : 0007 - ADMIN: Modificar contraseña}\label{sec:uc0}

\subsection{Flujo de eventos}

\subsubsection{Básico}

\begin{enumerate}
\item Paso 1
El administrador seleccionará la opción de modificar contraseña.
\item Paso 2
El sistema mostrará todos los usuarios.
\item Paso 3
El administrador seleccionará el usuario del cual quiere modificar la contraseña.
\item Paso 4
El administrador modificará la contraseña del usuario.
\item Paso 5
El sistema modificará la contraseña del usuario.

\end{enumerate}

\subsection{Precondiciones}
\begin{itemize}
\item No hay precondiciones.
\end{itemize}
\subsection{Postcondiciones}
\begin{itemize}
\item Se redirige al administrador al menú principal.
\end{itemize}



\section{Caso de uso : 0008 - ADMIN: Modificar ranking}\label{sec:uc0}

\subsection{Flujo de eventos}

\subsubsection{Básico}

\begin{enumerate}
\item Paso 1
El administrador seleccionará la opción de modificar ranking.
\item Paso 2
El sistema mostrará el ranking.
\item Paso 3
El administrador modificará el ranking.
\item Paso 4
El sistema actualizará el ranking.


\end{enumerate}

\subsection{Precondiciones}
\begin{itemize}
\item No hay precondiciones.
\end{itemize}
\subsection{Postcondiciones}
\begin{itemize}
\item Se redirige al administrador al menú principal.
\end{itemize}



\section{Caso de uso : 0009 - USUARIO: Modificar o eliminar datos}\label{sec:uc0}

\subsection{Flujo de eventos}

\subsubsection{Básico}

\begin{enumerate}
\item Paso 1
El usuario seleccionará la opción de modificar datos de la cuenta.
\item Paso 2
El sistema mostrará los datos del usuario.
\item Paso 3
El usuario modificará o eliminará aquellos datos que crea necesarios.
\item Paso 4
El sistema actualizará los nuevos datos o eliminará aquellos seleccionados.
\end{enumerate}


\subsection{Precondiciones}
\begin{itemize}
\item No hay precondiciones.
\end{itemize}
\subsection{Postcondiciones}
\begin{itemize}
\item Se redirige al usuario al menú principal.
\end{itemize}



\section{Caso de uso : 0010 - SISTEMA: Gestionar juego}\label{sec:uc0}

\subsection{Flujo de eventos}

\subsubsection{Básico}

\begin{enumerate}
\item El sistema se engarará de gestionar todo lo que está relacionado con el juego, tanto el ranking, como las diferentes cuentas de usuarios, etc.
\end{enumerate}

\subsection{Precondiciones}
\begin{itemize}
\item No hay precondiciones.
\end{itemize}
\subsection{Postcondiciones}
\begin{itemize}
\item No hay postcondiciones.
\end{itemize}



\section{Caso de uso : 0011 - SISTEMA: Gestionar niveles}\label{sec:uc0}

\subsection{Flujo de eventos}

\subsubsection{Básico}

\begin{enumerate}
\item El sistema se engarará de gestionar los diferentes niveles del juego, de menor a mayor dificultad.
\end{enumerate}

\subsection{Precondiciones}
\begin{itemize}
\item No hay precondiciones.
\end{itemize}
\subsection{Postcondiciones}
\begin{itemize}
\item No hay postcondiciones.
\end{itemize}



\section{Caso de uso : 0012 - ADMIN: Modificar o eliminar datos}\label{sec:uc0}

\subsection{Flujo de eventos}

\subsubsection{Básico}

\begin{enumerate}
\item Paso 1
El usuario seleccionará la opción de estados de los usuarios.
\item Paso 2
El sistema mostrará en que estado se encuentra cada usuario (jugando, esperando, etc.).
\end{enumerate}

\subsection{Precondiciones}
\begin{itemize}
\item No hay precondiciones.
\end{itemize}
\subsection{Postcondiciones}
\begin{itemize}
\item Se redirige al administrador al menú principal.
\end{itemize}



\section{Caso de uso : 0013 - USUARIO: Continuar partida}\label{sec:uc0}

\subsection{Flujo de eventos}

\subsubsection{Básico}

\begin{enumerate}
\item Paso 1
El usuario seleccionará la opción de continuar partida.
\item Paso 2
El sistema mostrará en que nivel se encuentra el usuario y su puntuación.
\item Paso 3
El usuario decide si quiere seguir con la partida, o iniciar una de nueva.
\end{enumerate}

\subsection{Precondiciones}
\begin{itemize}
\item No hay precondiciones.
\end{itemize}
\subsection{Postcondiciones}
\begin{itemize}
\item Se redirige al usuario a la partida anterior o a una nueva.
\end{itemize}





\section{Caso de uso : 0014 - USUARIO: Gestión de partidas existentes}\label{sec:uc0}

\subsection{Flujo de eventos}

\subsubsection{Básico}

\begin{enumerate}
\item Paso 1
El jugador tendrá en el menú la opción para gestionar sus partidas existentes.
\item Paso 2
El sistema mostrará una lista de las partidas existentes.
\end{enumerate}

\subsection{Precondiciones}
\begin{itemize}
\item No hay precondiciones.
\end{itemize}
\subsection{Postcondiciones}
\begin{itemize}
\item Se le enseña al usuario las partidas existentes. 
\end{itemize}



\section{Caso de uso : 0015 - USUARIO: Nueva partida}\label{sec:uc0}

\subsection{Flujo de eventos}

\subsubsection{Básico}

\begin{enumerate}
\item Paso 1
El usuario seleccionará la opción de partida nueva. 
\item Paso 2
El sistema permitirá al usuario iniciar una nueva partida sin que la anterior se borre (si la anterior se guarda). 
\end{enumerate}

\subsection{Precondiciones}
\begin{itemize}
\item No hay precondiciones.
\end{itemize}
\subsection{Postcondiciones}
\begin{itemize}
\item Se inicia una partida nueva y la anterior queda registrada. 
\end{itemize}




\section{Caso de uso : 0016 - USUARIO: Visualización del ranking}\label{sec:uc0}

\subsection{Flujo de eventos}

\subsubsection{Básico}

\begin{enumerate}
\item Paso 1
El usuario seleccinará la opción de ver los ranking. 
\item Paso 2
El sistema mostrará todos los rankings del jugador.
\item Paso 3
El usuario decide si quiere seguir jugando o salir.
\end{enumerate}

\subsection{Precondiciones}
\begin{itemize}
\item No hay precondiciones.
\end{itemize}
\subsection{Postcondiciones}
\begin{itemize}
\item Se le muestra al usuario los rankings acumulados. 
\end{itemize}



\section{Caso de uso : 0017 - USUARIO: Modificación de los ajustes}\label{sec:uc0}

\subsection{Flujo de eventos}

\subsubsection{Básico}

\begin{enumerate}
\item Paso 1
El usuario seleccionará la opción de modificar ajustes. 
\item Paso 2
El sistema permitirá al usuario hacer las modificaciones (de ajustes) que él crea convenientes. 
\item Paso 3
El usuario podrá hacer las modificaciones necesarias y decidirá si quiere jugar una partida o salir.
\end{enumerate}

\subsection{Precondiciones}
\begin{itemize}
\item No hay precondiciones.
\end{itemize}
\subsection{Postcondiciones}
\begin{itemize}
\item El usuario hace modificaciones de ajustes mediante una opción en el menú.
\end{itemize}


\section{Caso de uso : 0018 - USUARIO: Volver al menú principal}\label{sec:uc0}

\subsection{Flujo de eventos}

\subsubsection{Básico}

\begin{enumerate}
\item Paso 1
El usuario seleccionará la opción de menú principal. 
\item Paso 2
El sistema redireccionara al usuario al menú principal. 
\item Paso 3
El usuario podrá elegir alguna de las opciones existentes de este menú (el principal).
\end{enumerate}

\subsection{Precondiciones}
\begin{itemize}
\item No hay precondiciones.
\end{itemize}
\subsection{Postcondiciones}
\begin{itemize}
\item El usuario podrá acceder al menú principal.
\end{itemize}



\section{Caso de uso : 0019 - USUARIO: Salir del juego}\label{sec:uc0}

\subsection{Flujo de eventos}

\subsubsection{Básico}

\begin{enumerate}
\item Paso 1
El usuario seleccionará la opción de salir del juego. 
\item Paso 2
El juego se cerrará.
\end{enumerate}

\subsection{Precondiciones}
\begin{itemize}
\item No hay precondiciones.
\end{itemize}
\subsection{Postcondiciones}
\begin{itemize}
\item El usuario podrá salir del juego cuando quiera. 
\end{itemize}



\section{Caso de uso : 0020 - USUARIO: Guardado manual}\label{sec:uc0}

\subsection{Flujo de eventos}

\subsubsection{Básico}

\begin{enumerate}
\item Paso 1
El usuario guardará la partida manualmente.  
\item Paso 2
El sistema guardará la partida jugada hasta ese momento. 
\item Paso 3
El usuario escogerá si salir del juego o continuar jugando. 
\end{enumerate}

\subsection{Precondiciones}
\begin{itemize}
\item No hay precondiciones.
\end{itemize}
\subsection{Postcondiciones}
\begin{itemize}
\item El usuario podrá guardar la partida manualmente. 
\end{itemize}



\section{Caso de uso : 0021 - ADMIN: Borrar juego}\label{sec:uc0}

\subsection{Flujo de eventos}

\subsubsection{Básico}

\begin{enumerate}
\item Paso 1
El admin podrá borrar el juego como administrador. 
\item Paso 2
El sistema se borrará por completo. 
\end{enumerate}

\subsection{Precondiciones}
\begin{itemize}
\item No hay precondiciones.
\end{itemize}
\subsection{Postcondiciones}
\begin{itemize}
\item El administrador tendrá la opción de eliminar el juego.
\end{itemize}

\section{Caso de uso : 0022 - USUARIO: Comenzar partida de nuevo}\label{sec:uc0}

\subsection{Flujo de eventos}

\subsubsection{Básico}

\begin{enumerate}
\item Paso 1
El usuario tendrá opción de empezar la partida de nuevo. 
\item Paso 2
El sistema se actualizará con una nueva partida. 
\end{enumerate}

\subsection{Precondiciones}
\begin{itemize}
\item No hay precondiciones.
\end{itemize}
\subsection{Postcondiciones}
\begin{itemize}
\item El usuario empezará la misma partida de nuevo si desea. 
\end{itemize}