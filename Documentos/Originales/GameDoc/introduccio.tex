\section{Introducción}\label{sec:intro}


La idea principal del juego está basada en un típico Escape Room. Los usuarios deberán registrarse y tendrán que irse pasando las diferentes pruebas que se irán encontrando en las habitaciones (las cuales corresponden a los niveles del juego). Cuando se complete cada nivel, se podrá acceder al siguiente. Para añadir un nivel de competitividad al juego, este incluye un ranking que acumulará los mejores tiempos y el mayor número de niveles superados por cada usuario. En los diferentes niveles (habitaciones) se pueden obtener de forma aleatoria ciertos objetivos o “bonus” que posteriormente permitirán a los usuarios saltarse niveles o recibir pistas, e incluso reducir el tiempo final de la prueba entre otros.
Dentro de los diferentes niveles, se encuentran los niveles normales, los cuales se centran únicamente en superar la prueba correspondiente, y los niveles contrarreloj, que son aleatorios y que obligan al jugador a pasarse dicho nivel en un tiempo establecido. En caso de no conseguirlo, la partida terminará y se deberá volver a empezar desde el inicio.
El juego es individual y no incluye ningún elemento multijugador.
No existe una forma de ser eliminado a excepción de los niveles contrareloj. 
Las diferentes puntuaciones del ranking son individuales, sin embargo, estas se almacenan de forma online, y todo el mundo puede acceder a ellas desde el ranking y consultar los resultados de los demás. 
El progreso de los usuarios se puede guardar, para posteriormente continuar por donde dejaron la partida. Esto implica también que se quedará registrado el tiempo que lleva dicho usuario hasta ahora y los niveles que ha superado.
En el juego existe un rol de administrador que cuenta con ciertos privilegios, como poder gestionar los niveles y administrar las cuentas de los usuarios registrados.


\begin{figure}[ht]
  \centering
  \includegraphics[width=0.6\textwidth]{./imatges/logo.png}
  \caption{Logo del juego}
  \label{fig:gamelogo}
\end{figure}
