\section{Històries}

Se han recogido todas las necesidades para llevar a cabo el desarrollo.

-	El usuario debe conectarse al sistema mediante un usuario y una contraseña.

-	Para pasar de nivel los usuarios deben completar los objetivos del nivel.

-	El juego tiene un ranking que se podrá consultar desde internet.

-	En las habitaciones se pueden encontrar objetivos o “bonus” que ayudarán al usuario a pasarse los posteriores niveles o acceder a los siguientes.

-	En algunos niveles aparecen pruebas contrarreloj que obligan al usuario a pasarlo en tiempo o serán eliminados cuando finalice el tiempo establecido por el nivel.

-	Si el jugador es eliminado en una prueba contrareloj, debe volver a empezar la partida desde el principio.

-	El juego tiene los roles de administrador y de jugador.

-	El administrador puede crear y editar nuevos mapas.

-	El administrador puede modificar la contraseña del usuario.

-	El administrador puede eliminar la cuenta del usuario.

-	El administrador puede modificar el ranking de los usuarios.

-	El usuario puede modificar o eliminar sus datos de la cuenta.

-	El sistema se encarga de gestionar todo lo relacionado con el juego (rankings, cuentas, niveles...).

-	El sistema es el encargado de generar los mapas aleatoriamente y sus propiedades (tipo de salas o habitaciones, la dificultad del nivel, tipo de nivel: normal o contrarreloj).

-	Hay cierta cantidad de mapas de cada dificultad, que son seleccionados por el sistema de forma aleatoria en función del nivel en el que te encuentras.

-	El sistema gestiona los niveles de menor a mayor dificultad.

-	El administrador puede ver el estado de los usuarios que se encuentran en linea (en partida...).

-	El administrador puede crear cuentas nuevas.

-	El jugador puede continuar la partida por donde la dejó.

-	El sistema se encarga de guardar el estado de la partida del usuario.

-	El jugador puede empezar una partida nueva.

-	El jugador puede borrar una partida existente.

-	El jugador puede consultar a los bonus obtenidos anteriormente, que serán mostrados por pantalla y puede seleccionar entre utilizar uno de ellos o no usar ninguno.

-	En la pantalla se mostrará al jugador si se encuentra en una pantalla contrarreloj y el tiempo del cual dispone para pasarse el nivel.

-	En cada turno el jugador puede consultar todas las acciones que tiene disponibles en la propia pantalla (menú de movimientos).

-	El jugador puede guardar la partida manualmente desde el menú de la partida.

-	El jugador tiene en el menú de la partida la opción para salir del juego.

-	El jugador tiene en el menú de la partida la opción para volver al menú principal.

-	El jugador tiene en el menú principal la opción para modificar sus ajustes.

-	El jugador tiene en el menú principal la opción para ver los rankings.

-	El jugador tiene en el menú principal la opción para empezar una partida nueva.

-	El jugador tiene en el menú principal la opción para gestionar sus partidas existentes.

-         El jugador puede acceder a los movimientos que ha realizado previamente durante la partida.

-	El jugador finalizará la partida si completa todos los niveles existentes en la partida.

